\section{Define the methodology and lifecycle}\label{sec:define-the-methodology-and-lifecycle}

\subsection{Methodology: How do we carry out this project?}\label{subsec:methodology:how-do-we-carry-out-this-project}

\begin{itemize}
    \item TDSP (Scrum): It uses agile principles and team organization.
    It’s not a straightforward delivery, but a process of trial and error with different iterations, involving six months of testing and refining.
    A key aspect is the Customer Acceptance, which is essential in a medical context and must be revised continuously.
    TDSP also defines the roles within the team.
\end{itemize}

\subsection{Steps}\label{subsec:steps}

\begin{itemize}
    \item Understand the business: Conduct regular meetings with stakeholders, ideally once every two weeks, to gather requirements and align our problem definition.
    \item Modeling: Develop models based on the defined problem and requirements.
    \item Deployment: Implement the models in production.
    \item Iteration: Continuously refine and iterate the process.
    \item Customer Acceptance: Assess whether medical professionals agree with the model.
    Feedback is critical throughout the project.
\end{itemize}

\subsection{Practical Approach:}\label{subsec:practical-approach:}

\begin{itemize}
    \item Start with a representative hospital that has implemented integrated systems between laboratories (using a test set as the first approach).
    \item Work with the same group of doctors for consistent feedback or compare results with similar hospitals, such as hospitals within a specific city.
    \item Ensure patient data confidentiality is maintained at all times.
\end{itemize}

\subsection{Ethical Considerations}\label{subsec:ethical-considerations}

\begin{itemize}
    \item Introduce an ethics committee, external to the team, that reviews every project delivery.
    \item Ensure all steps are well-documented and verifiable.
\end{itemize}

\subsection{Lifecycle}\label{subsec:lifecycle}

\begin{itemize}
    \item Data Exploration (EDA): Analyze relationships, detect outliers, process and clean the data.
    \item MVP (Minimum Viable Product): Develop a model capable of making predictions.
    Focus on three specific diseases, the ones recommended by stakeholders.
    Consider key characteristics specific to certain diseases, such as hereditary factors.
    For example, consider cases where parents have a history of chronic illnesses to reduce the margin of error.
    \item Anonymization of Data: Use cryptographic techniques or add random noise to protect sensitive personal data.
    Later, sensitive information can be de-anonymized if necessary.
    Ensure protection of sensitive data, such as religion, politics, and health, in compliance with the EU Data Protection Regulation (GDPR), as these require special security.

\end{itemize}

\subsection{Ethics Committee Review}\label{subsec:ethics-committee-review}
Help prevent bias or the marginalization of minorities.
Before showing proposed approaches to the ethics committee, ensure compliance with European regulations.
Also consult legal and privacy experts on matters of transparency and data protection.

\subsection{Additional consideration}\label{subsec:additional-consideration}

Using Microsoft tools (as specified by the TDSP guidelines) is very important in the context of a public organization, since they will most likely already be using Microsoft tools.

\subsection*{}
\begin{warning}
    \textbf{Challenge}: Oops!
    The Data Science Team noticed that the data is biased.

    \textbf{Solution}: Collect more data from other hospitals to counteract the imbalance in the dataset.
    Gather the data from hospitals or datasets where there’s a higher diversity of patients (socially, economically…)


\end{warning}