\section{Communication strategy}\label{sec:communication-strategy}

Let's divide the communication strategy into two parts:

\begin{enumerate}
    \item Communication during product development and it handover to clients


    \item Communication after the project release.


\end{enumerate}
At the same time, all communication between us and the stakeholders will take place through two representatives:

\begin{enumerate}
    \item The Project Manager (or his\\her deputy) - regulatory and organizational issues


    \item The Project Technical Lead (or his\\her deputy) - technical issues


\end{enumerate}
Meetings and negotiations with stakeholders will be held either online or in person at a location chosen by us or the stakeholders.
Preference, of course, will be given to live meetings.

The main communication tools will be Teams and MS Planner.
In special cases, email may be used, but it will be mandatory to mention it in Planner as well.

For each stakeholder group, a user will be created in our project management system.
Obviously, they will only have access to tasks related to their area of expertise.
The tasks for stakeholders will primarily consist of the following requests:

\begin{itemize}
    \item Providing documentation regarding input data


    \item Additional clarifications about the physics of processes that can enhance the accuracy and reliability of the developed model.


\end{itemize}
Meeting reports will also be published in the project management system.

During the active development phase of the project, a meeting schedule will be agreed upon with the stakeholders.
These meetings will include presentations of the project's current progress, collection of recommendations, discussions of potential improvements, and so on.

However, the final meetings at the time of handing over the completed project to the clients will be held in person with representatives from each side.

A schedule and procedure for delivering the completed product to the consumers will also be agreed upon.
It will include deadlines, key milestones, and the components of the project that will be handed over.

During the project handover, our team will include both technical and organizational representatives.
Typically, this includes the project manager, solution architect, technical lead, and a group of developers for installing the developed software complex on the client's hardware.

After the project handover, each client will be granted access to the technical and user support system.
Any subsequent questions regarding the operation of our product will be discussed there.

All communication with patients will take place only with the involvement of medical experts.

Simplified diagrams of the model's operation, along with a list of the principles we have already mentioned in our solution for the ethical task, will also be prepared for patients (i.e., the general public).

Based on the principles we outlined in the ethical section, a memo will be developed to explain our approaches when working with patients' personal data.

This memo will be published in the institutions using our development and will be freely accessible to potential and current clients of our product.

The use of a patient's data in the model is only possible with their consent.
Each patient will be offered by their attending physician to participate in the training of our model.

Channel to check personal data used, see its uses, withdraw permission. \\

Comments by one professor:

\begin{itemize}
    \item Give tools so that the doctors can communicate model results to the patients (give short phrases… etc)


    \item Give the same disclaimer tools to hospital public faces


\end{itemize}

\subsection*{}
\begin{warning}
    \textbf{Challenge}: The stakeholders brought a top American consultant to sit in on the presentation, and she only speaks Murican.

    \textbf{Solution}: For this challenge we will simply speak in English with the American consultant.


\end{warning}





