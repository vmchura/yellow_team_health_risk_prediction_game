\section{Ethical considerations}\label{sec:ethical-considerations}

We’ll summarize the considerations we’ve discussed throughout the project:

\begin{itemize}
    \item False Negatives: As mentioned in the model section, we’ll prioritize minimizing false negatives.
    In a medical context, generating a false negative is more critical than aiming for perfect accuracy.


    \item Ethics Committee: We’ll involve a group of external specialists to guide us through doubts and decisions,
    as well as to validate the project.
    At a minimum, they will conduct a final validation, but given the project’s sensitivity,
    we aim to involve them at the start and at intermediate milestones.


    \item Confidentiality: We’ll work exclusively with anonymized data,
    leveraging patient medical histories and their evolution over time.
    Only doctors will have access to this information via the application.
    Access to the server and database will be limited to one person with private keys.
    Additionally, any data we decide to collect will be validated with the ethics committee to ensure they are sufficiently anonymized and feasible for training a model.


    \item Responsibility and Licensing:
    All professionals using the tool must go through a licensing system and accept terms of use aligned with the ethical document and established conditions.
    For example, the tool can only be used by authorized individuals or companies, solely for medical purposes,
    and not for discriminating against individuals for non-health-related goals.
    We’ll rely on the application’s administrators to advise us on repercussions and actions to take against those who breach these terms.


    \begin{itemize}
        \item For final patients, we can prepare an explanatory video and ask them to read a usage guide for data.


    \end{itemize}
    \item Biases: We’ll make every effort to detect biases at any point in the project,
    leveraging data from other hospitals if we detect anomalies or disparities, as mentioned in previous challenges.


    \item Output and Transparency: The model will not provide suggestions unless explicitly requested by the doctor.
    Using an ensemble model, we’ll provide technical outputs (as specified by doctors for each condition,
    e.g., glucose level, heart rate) to ensure the model remains explainable and transparent.

\end{itemize}

Comments by one professor:
\begin{itemize}
    \item Governance and Accountability Report: There should be a clear report detailing the governance and accountability of each part of the process, including data and who is responsible for what.


    \item Patient Consent: The final patient must have the ability to withdraw consent for sharing their data at any time. They should be able to both give and revoke consent easily.


    \item Continuous Monitoring: As already included in the budgets for ongoing monitoring, this aspect should be explicitly mentioned here as a critical part of the process.


    \item Data Minimization: Only use the data strictly necessary for the model, ensuring we’re not working with excessive or irrelevant information.


\end{itemize}

\subsection*{}
\begin{warning}
    \textbf{Challenge}: A team you depend on will not provide the data on time.

    \textbf{Solution}: We assume that it’s an external medical team.

    \begin{itemize}
        \item Change the order of illnesses handling of the project.
        For example if we can’t handle diabetes at the start, try to see if we have enough data on another disease.
        We should agree on the order we will handle each illness and schedule it, to ensure this issue doesn’t happen again,
        and so that they commit to having the data ready on time.
        It would be ideal if the Project Manager could oversee that we have the data on time.


        \item If possible and if the stakeholders agree, look for external data to replace the currently missing data
        (temporarily or permanently) in similar organizations.
        Of course, whether the data from another hospital is suitable as a replacement is up to the medical experts to decide.


        \item Agree that each deadline will be relative to the date we receive the necessary data to begin.


    \end{itemize}


\end{warning}